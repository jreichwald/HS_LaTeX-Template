% Template created by 
% Prof. Dr. Julian Reichwald, j.reichwald@hs-mannheim.de 
%
\documentclass[
	12pt,
	BCOR=5mm,
	DIV=12,
	headinclude=on,
	footinclude=off,
	parskip=half,
	bibliography=totoc,
	listof=entryprefix,
	toc=listof,
	numbers=noenddot,
	plainfootsepline]{scrreprt}

%	Konfigurationsdatei einziehen
% !TEX root =  master.tex

%		LANGUAGE SETTINGS AND FONT ENCODING 
%
\usepackage[ngerman]{babel} 	% German language
\usepackage[utf8]{inputenc}
\usepackage[german=quotes]{csquotes} 	% correct quotes using \enquote{}
\usepackage[T1]{fontenc}


\usepackage{enumitem}
\setlist[itemize]{leftmargin=*}
\setlist[enumerate]{leftmargin=*}

%\usepackage[english]{babel}   % For english language
%\usepackage{csquotes} 	% Richtiges Setzen der Anführungszeichen mit \enquote{}

% 		HYPERREF
%
\usepackage[
	hidelinks=true % keine roten Markierungen bei Links
]{hyperref}

% Zwei eigene Befehle zum Setzen von Autor und Titel. Ausserdem werden die PDF-Informationen richtig gesetzt.
\newcommand{\ThesisTitle}[1]{\def\TheThesisTitle{#1}\hypersetup{pdftitle={#1}}}
\newcommand{\ThesisAuthor}[1]{\def\TheThesisAuthor{#1}\hypersetup{pdfauthor={#1}}}

% Correct superscripts 
\usepackage{fnpct}




%		CALCULATIONS
%
\usepackage{calc} % Used for extra space below footsepline



%		BIBLIOGRAPHY SETTINGS
%

% Uncomment the next three lines for author-year-style with footnotes (Chicago)
%\usepackage[backend=biber, autocite=footnote, style=authoryear, dashed=false]{biblatex} 	%Use Author-Year-Cites with footnotes
%\AdaptNoteOpt\footcite\multfootcite   %will add  separators if footcite is called multiple consecutive times 
%\AdaptNoteOpt\autocite\multautocite % will add  separators if autocite is called multiple consecutive times

% Uncomment the next line for IEEE-style 
\usepackage[backend=biber, autocite=inline, style=ieee]{biblatex} 	% Use IEEE-Style (e.g. [1])

% Uncomment the next line for alphabetic style 
% \usepackage[backend=biber, autocite=inline, style=alphabetic]{biblatex} 	% Use alphabetic style (e.g. [TGK12])

% Uncomment the next two lines vor Harvard-Style 
%\usepackage[backend=biber, style=apa]{biblatex} 	
%\DeclareLanguageMapping{german}{german-apa}


\DefineBibliographyStrings{ngerman}{  %Change u.a. to et al. (german only!)
	andothers = {{et\,al\adddot}},
}

\renewcommand*{\mkbibacro}[1]{#1} % Use same font for URL/DOI/ISBN in bibliography 
\urlstyle{same} % use same font for DOI- and URL-Links 

%%% Uncomment the following lines to support hard URL breaks in bibliography 
%\apptocmd{\UrlBreaks}{\do\f\do\m}{}{}
%\setcounter{biburllcpenalty}{9000}% Kleinbuchstaben
%\setcounter{biburlucpenalty}{9000}% Großbuchstaben


\setlength{\bibparsep}{\parskip}		%add some space between biblatex entries in the bibliography
\addbibresource{bibliography.bib}	%Add file bibliography.bib as biblatex resource


%		FOOTNOTES 
%
% Count footnotes over chapters
\usepackage{chngcntr}
\counterwithout{footnote}{chapter}

%	ACRONYMS
%%%
%%% WICHTIG: Installieren Sie das neueste Acronyms-Paket!!!
%%%
\makeatletter
\usepackage[printonlyused]{acronym}
\@ifpackagelater{acronym}{2015/03/20}
  {%
    \renewcommand*{\aclabelfont}[1]{\textbf{{\acsfont{#1}}}}
  }%
  {%
  }%
\makeatother

%		LISTINGS
\usepackage{listings}	%Format Listings properly
\renewcommand{\lstlistingname}{Quelltext} 
\renewcommand{\lstlistlistingname}{Quelltextverzeichnis}
\lstset{numbers=left,
	numberstyle=\tiny,
	captionpos=b,
	basicstyle=\ttfamily\small}


%		EXTRA PACKAGES
\usepackage{blindtext}    %Blindtext
\usepackage{graphicx} % use various graphics formats
\usepackage[german]{varioref} 	% nicer references \vref
\usepackage{caption}	%better Captions
\usepackage{booktabs} %nicer Tabs
\usepackage{array}


%		ALGORITHMS
\usepackage{algorithm}
\usepackage{algpseudocode}
\renewcommand{\listalgorithmname}{Algorithmenverzeichnis }
\floatname{algorithm}{Algorithmus}


%		FONT SELECTION: Entweder Latin Modern oder Times / Helvetica
%\usepackage{lmodern} %Latin modern font

%\usepackage{mathptmx}  %Helvetica / Times New Roman fonts (2 lines)
%\usepackage[scaled=.92]{helvet} %Helvetica / Times New Roman fonts (2 lines)
\usepackage{courier}   % Courier for monospaced fonts
\usepackage{fourier}   % Fourier/Utopia for rm text 
\addtokomafont{disposition}{\rmfamily} % Use serif-font for section headers
\setkomafont{descriptionlabel}{\normalfont\bfseries} % Use bold serif-font for description labels

%		PAGE HEADER / FOOTER
%	    Warning: There are some redefinitions throughout the master.tex-file!  DON'T CHANGE THESE REDEFINITIONS!
\RequirePackage[automark,headsepline,footsepline]{scrlayer-scrpage}
\pagestyle{scrheadings}

% Use e.g. \renewcommand*{\pnumfont}{\upshape\sffamily} for Sans Serif font in page header and footer  
\renewcommand*{\pnumfont}{\upshape}
\renewcommand*{\headfont}{\upshape}
\renewcommand*{\footfont}{\upshape}
\renewcommand{\chaptermarkformat}{}
\RedeclareSectionCommand[beforeskip=0pt]{chapter}
\clearscrheadfoot

\ifoot[\rule{0pt}{\ht\strutbox+\dp\strutbox}HS Mannheim]{\rule{0pt}{\ht\strutbox+\dp\strutbox}HS Mannheim}
\ofoot[\rule{0pt}{\ht\strutbox+\dp\strutbox}\pagemark]{\rule{0pt}{\ht\strutbox+\dp\strutbox}\pagemark}

\ohead{\headmark}


\begin{document}

%% BITTE GEBEN SIE HIER DEN TITEL UND DIE AUTORIN / DEN AUTOR DER ARBEIT AN!
%% DIESE INFORMATIONEN _MÜSSEN_ GESETZT SEIN, UM TITELBLATT, ABSTRACT UND
%% EIGENSTÄNDIGKEITSERKLÄRUNG AUTOMATISCH ANZUPASSEN!
\ThesisTitle{Entwurf und Implementierung eines verteilten, direkten Suchverfahrens als beispielhafter Titel einer Arbeit}
\ThesisAuthor{Erika Musterfrau}

%!TEX root =  master.tex
\begin{titlepage}
		\vspace{-2cm}
		\centering\noindent\includegraphics[scale=1.0]{img/thmannheimlogo.pdf}

\vspace{3em}

\begin{center}
	{\textbf{\Large{}\TheThesisTitle}}\\[5em]
	{\textbf{\large{}Bachelorarbeit}}\\[2em]
	{\textbf{Fakultät für Wirtschaftsingenieurwesen}\\[.5em] \textbf{Hochschule Mannheim}}
	
	\vspace{3em}
\vfill

\begin{minipage}{\textwidth}

\begin{tabbing}
	Zweitkorrektor*in \hspace{0.85cm}\=\kill
	Verfasser/in: \> \TheThesisAuthor \\[1.5mm]
	Matrikelnummer: \> 123456 \\[1.5mm]
	Unternehmen: \> Das Unternehmen, falls vorhanden (sonst bitte streichen)  \\[1.5mm]
	Erstkorrektor*in: \> Prof. Dr. Max Mustermann \\[1.5mm]
	Zweitkorrektor*in: \> Prof. Dr. Max Mustermann \\[1.5mm]

\end{tabbing}
\end{minipage}

\end{center}

\end{titlepage}

\pagenumbering{roman} % Römische Seitennummerierung
\normalfont

%--------------------------------
% Verzeichnisse - nicht benötige Verzeichnisse bitte auskommentieren / löschen.
%--------------------------------

%   Sperrvermerk
\input{nondisclosurenotice}

%	Kurzfassung
\chapter*{Kurzfassung}
\begingroup
\begin{table}[h!]
\setlength\tabcolsep{0pt}
\begin{tabular}{p{3.7cm}p{11.7cm}}
Titel & \TheThesisTitle \\
Verfasser/in: & \TheThesisAuthor \\
\end{tabular}
\end{table}
\endgroup

Hier können Sie die Kurzfassung der Arbeit schreiben. Diese sollte -- wie der Name schon sagt -- \textit{kurz} sein.




%	Inhaltsverzeichnis
\tableofcontents

%	Abbildungsverzeichnis
\listoffigures

%	Tabellenverzeichnis
\listoftables

%	Listingsverzeichnis
 \lstlistoflistings

% 	Algorithmenverzeichnis
\listofalgorithms

% 	Abkürzungsverzeichnis (siehe Datei acronyms.tex!)
\clearpage
\chapter*{Abkürzungsverzeichnis}	
\addcontentsline{toc}{chapter}{Abkürzungsverzeichnis}


\begin{acronym}[RDBMS]
	\acro{RDBMS}{Relational Database Management System}
	\acro{BMBF}{Bundesministerium für Bildung und Forschung}	
\end{acronym}

\ohead{Acronyms} % Neue Header-Definition

%--------------------------------
% Start des Textteils der Arbeit
%--------------------------------
\clearpage
\ihead{\chaptername~\thechapter} % Neue Header-Definition (inner header)
\ohead{\headmark} % Neue Header-Definition (outer header)
\pagenumbering{arabic}  % Arabische Seitenzahlen


% Blindtext, um das Layout zu testen 
\Blinddocument
\section{Formelsatz}
\blindmathpaper

% Anleitungs-Datei anleitung.tex einziehen. Auf diese Weise sollten Sie versuchen, für jedes einzelne
% Kapitel eine eigene Datei anzulegen und mittels input-Kommando einzuziehen.
\input{anleitung}


%	Literaturverzeichnis
\clearpage
\ihead{}
\printbibliography[title=Literaturverzeichnis]
\cleardoublepage

% Der Anhang beginnt hier - jedes Kapitel wird alphabetisch aufgezählt. (Anhang A, B usw.)
\appendix
\ihead{\appendixname~\thechapter} % Neue Header-Definition

% appendix.tex einziehen
\chapter{Testanhang}
\blindtext



\section{Subtestanhang}

\chapter{Noch ein Testanhang}



% Ehrenwörtliche Erklärung ewerkl.tex einziehen
\input{ewerkl.tex}


\end{document}
