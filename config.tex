% !TEX root =  master.tex

%		LANGUAGE SETTINGS AND FONT ENCODING 
%
\usepackage[ngerman]{babel} 	% German language
\usepackage[utf8]{inputenc}
\usepackage[german=quotes]{csquotes} 	% correct quotes using \enquote{}
\usepackage[T1]{fontenc}


\usepackage{enumitem}
\setlist[itemize]{leftmargin=*}
\setlist[enumerate]{leftmargin=*}

%\usepackage[english]{babel}   % For english language
%\usepackage{csquotes} 	% Richtiges Setzen der Anführungszeichen mit \enquote{}

% 		HYPERREF
%
\usepackage[
	hidelinks=true % keine roten Markierungen bei Links
]{hyperref}

% Zwei eigene Befehle zum Setzen von Autor und Titel. Ausserdem werden die PDF-Informationen richtig gesetzt.
\newcommand{\ThesisTitle}[1]{\def\TheThesisTitle{#1}\hypersetup{pdftitle={#1}}}
\newcommand{\ThesisAuthor}[1]{\def\TheThesisAuthor{#1}\hypersetup{pdfauthor={#1}}}

% Correct superscripts 
\usepackage{fnpct}




%		CALCULATIONS
%
\usepackage{calc} % Used for extra space below footsepline



%		BIBLIOGRAPHY SETTINGS
%


% Uncomment the next three lines for author-year-style with footnotes (Chicago)
\usepackage[backend=biber, autocite=footnote, style=authoryear, dashed=false]{biblatex} 	%Use Author-Year-Cites with footnotes

% Uncomment the next line for IEEE-style 
%\usepackage[backend=biber, autocite=inline, style=ieee]{biblatex} 	% Use IEEE-Style (e.g. [1])

% Uncomment the next line for alphabetic style 
% \usepackage[backend=biber, autocite=inline, style=alphabetic]{biblatex} 	% Use alphabetic style (e.g. [TGK12])

% Uncomment the next two lines vor Harvard-Style 
% \usepackage[backend=biber, style=apa]{biblatex} 	
% \DeclareLanguageMapping{german}{german-apa}


\DefineBibliographyStrings{ngerman}{  %Change u.a. to et al. (german only!)
	andothers = {{et\,al\adddot}},
}

\renewcommand*{\mkbibacro}[1]{#1} % Use same font for URL/DOI/ISBN in bibliography 
\urlstyle{same} % use same font for DOI- and URL-Links 

%%% Uncomment the following lines to support hard URL breaks in bibliography 
%\apptocmd{\UrlBreaks}{\do\f\do\m}{}{}
%\setcounter{biburllcpenalty}{9000}% Kleinbuchstaben
%\setcounter{biburlucpenalty}{9000}% Großbuchstaben


\setlength{\bibparsep}{\parskip}		%add some space between biblatex entries in the bibliography
\addbibresource{bibliography.bib}	%Add file bibliography.bib as biblatex resource


%		FOOTNOTES 
%
% Count footnotes over chapters
\usepackage{chngcntr}
\counterwithout{footnote}{chapter}

%	ACRONYMS
%%%
%%% WICHTIG: Installieren Sie das neueste Acronyms-Paket!!!
%%%
\makeatletter
\usepackage[printonlyused]{acronym}
\@ifpackagelater{acronym}{2015/03/20}
  {%
    \renewcommand*{\aclabelfont}[1]{\textbf{{\acsfont{#1}}}}
  }%
  {%
  }%
\makeatother

%		LISTINGS
\usepackage{listings}	%Format Listings properly
\renewcommand{\lstlistingname}{Quelltext} 
\renewcommand{\lstlistlistingname}{Quelltextverzeichnis}
\lstset{numbers=left,
	numberstyle=\tiny,
	captionpos=b,
	basicstyle=\ttfamily\small}


%		EXTRA PACKAGES
\usepackage{blindtext}    %Blindtext
\usepackage{graphicx} % use various graphics formats
\usepackage[german]{varioref} 	% nicer references \vref
\usepackage{caption}	%better Captions
\usepackage{booktabs} %nicer Tabs
\usepackage{array}


%		ALGORITHMS
\usepackage{algorithm}
\usepackage{algpseudocode}
\renewcommand{\listalgorithmname}{Algorithmenverzeichnis }
\floatname{algorithm}{Algorithmus}


%		FONT SELECTION: Entweder Latin Modern oder Times / Helvetica
%\usepackage{lmodern} %Latin modern font

%\usepackage{mathptmx}  %Helvetica / Times New Roman fonts (2 lines)
%\usepackage[scaled=.92]{helvet} %Helvetica / Times New Roman fonts (2 lines)
\usepackage{courier}   % Courier for monospaced fonts
\usepackage{fourier}   % Fourier/Utopia for rm text 
\addtokomafont{disposition}{\rmfamily} % Use serif-font for section headers
\setkomafont{descriptionlabel}{\normalfont\bfseries} % Use bold serif-font for description labels

%		PAGE HEADER / FOOTER
%	    Warning: There are some redefinitions throughout the master.tex-file!  DON'T CHANGE THESE REDEFINITIONS!
\RequirePackage[automark,headsepline,footsepline]{scrlayer-scrpage}
\pagestyle{scrheadings}

% Use e.g. \renewcommand*{\pnumfont}{\upshape\sffamily} for Sans Serif font in page header and footer  
\renewcommand*{\pnumfont}{\upshape}
\renewcommand*{\headfont}{\upshape}
\renewcommand*{\footfont}{\upshape}
\renewcommand{\chaptermarkformat}{}
\RedeclareSectionCommand[beforeskip=0pt]{chapter}
\clearscrheadfoot

\ifoot[\rule{0pt}{\ht\strutbox+\dp\strutbox}HS Mannheim]{\rule{0pt}{\ht\strutbox+\dp\strutbox}HS Mannheim}
\ofoot[\rule{0pt}{\ht\strutbox+\dp\strutbox}\pagemark]{\rule{0pt}{\ht\strutbox+\dp\strutbox}\pagemark}

\ohead{\headmark}
